\problemset{Additional homework}

\textit{Исправлено: № 3, № 4.}

~\

~\

~\

\begin{problem}{1}

\end{problem}

~\

~\

\begin{solution}

Если L - полный язык для PH, то $L \in \Sigma_i^p$ для некоторого i.

Возьмем язык $L' \in PH$, он сводится к L, значит, $L' \in \Sigma_i^p$

\textit{Пояснение}

Что значит сводится? Есть полиномиальное сведение $x -> p(x)$

То есть для $L'$ верно

$x \in L' \Leftrightarrow \exists x_1 \forall x_2 ... x_i : M'(p(x), x_1, ..., x_i) = 1$

\end{solution}

~\

~\

\begin{problem}{2}


\end{problem}

~\

~\

\begin{solution}

$BP \cdot NP = \{L : L \leqslant_R 3SAT\}$

Во-первых, $3SAT \in NP \Rightarrow x \in 3SAT \Leftrightarrow \exists u : M_{3SAT}(x, u) = 1$

Во-вторых, $BPP \in \Sigma_2^p$, то есть для языка $B \in BPP$ верно $x \in B \Leftrightarrow \exists u \forall v M_B(x, u, v) = 1$

Мы рассматриваем множеств языков, сводимых к 3-САТУ некоторым алгоритмом B из BPP.

$M_B$ преобразует вход для 3-САТА и запускает его.

Таким образом $L \in BP \cdot NP \Rightarrow \exists u \forall v \exists u' M_{3SAT}(x, u, v, u') = 1$, то есть $L \in \Sigma_3^p$, то есть $BP \cdot NP \subset \Sigma_3^p$

\end{solution}

~\

~\


\begin{problem}{4}


\end{problem}

~\

~\

\begin{solution}

Пусть L разрешим, тогда сушествует машина M, которая его решает.

Попробуем решить HALT.

Заведем машину Тьюринга A.

Пусть $A_{x \: i}(t) = 0$, если $M_i$ на входе x работает более, чем за время t.

Иначе заставим A работать $100|t|^2 + 201$ шагов

Таким образом вот алгоритм решения HALT.

Принимаем на вход (i, x), i - номер машины, x - вход.

Запускаем $M(A_{x \: i})$. Если $A_{x \: i}$ останавливается до $100n^2 + 200$ шагов на всех входах, значит $M_i$ не останавливается на иксе никогда. Если все же существует t, такое, что $M_i$ останавливается за время t, то мы об этом узнаем, потому что найдется вход $A_{x \: i}$, на котором она будет работать долго.

Получается, мы решили HALT, противоречие

\end{solution}

~\

~\

\begin{problem}{5}

\end{problem}



\begin{solution}

Докажем по индукции, что $(\Sigma_i^p)^A = P^A$

База ясна: $(\Sigma_0^p)^A = P^A = (\Pi_0^p)^A$

Пусть $(\Sigma_i^p)^A = P^A = (\Pi_i^p)^A$ по предположению индукции

Докажем, что $(\Sigma_{i + 1}^p)^A = P^A$

$L \in \Sigma_{i + 1}^p \Leftrightarrow$

$x \in L \Leftrightarrow \exists u \in P(x) : M(x, u) = 1 $, где $M \in \Pi_i^p$

Значит,

$L \in (\Sigma_{i + 1}^p)^A \Leftrightarrow$

$x \in L \Leftrightarrow \exists u \in P^A(x) : M(x, u) = 1 $, где $M \in (\Pi_i^p)^A = P^A$, то есть $(\Sigma_{i + 1}^p)^A = (NP)^A = P^A$

$PH = \bigcup{\Sigma_i^p} \Rightarrow PH^A = P^A$


\end{solution}





