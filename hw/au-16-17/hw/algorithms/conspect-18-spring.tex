\documentclass{article}

\usepackage[T2A]{fontenc}           
\usepackage[utf8]{inputenc}         
\usepackage[russian,english]{babel} 
\usepackage{amsmath,amssymb}        
\usepackage{listings}
\usepackage{cmll}
\usepackage[left=15mm, top=1cm, right=15mm, bottom=1cm, nohead, footskip=1cm]{geometry}
\usepackage{fontspec}
\usepackage[usenames]{color}
\usepackage{colortbl} 
\usepackage{misccorr} 
\setmainfont{Times New Roman}
 
\newcommand{\problem}[1]{\textbf{Problem #1} \newline}
  
\newcommand{\solution}{\textbf{Solution} \newline}
 
\newcommand{\question}[2]{
	\begin{center}
		\large{\textbf{#1 (#2)}} 
	\end{center}
}
 
\begin{document}

\begin{center}

\Huge Algorithms

\end{center}

~\

~\


\begin{question}{Философский смысл класса. Метод борьбы с NP-трудностью задачи}{14}

\end{question}

~\

~\

\begin{question}{Метод диагонализации. Теорема об иерархии по времени.}{15}

\textbf{Теорема}

Пусть $f(n) \log{f(n)} = o(g(n))$. Тогда $DTime(f(n)) \subsetneq DTime(g(n))$

Берем машину D, которая симулирует $M_x(x)$, но если работает больше $n^{1.4}$, то останавливается и возвращает 0. Иначе возвращает инвертированный ответ. Тогда пусть равенство есть, возьмем номер машины M, которая работает за f, такой, что D на нем отработает и вернет инвертированный ответ, противоречие.


\end{question}

~\

~\

\begin{question}{}{16}

\end{question}

~\

~\

\begin{question}{}{16}

\end{question}

~\

~\

\begin{question}{}{16}

\end{question}

~\

~\


\end{document}
