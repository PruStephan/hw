\documentclass{article}

\usepackage[T2A]{fontenc}           
\usepackage[utf8]{inputenc}         
\usepackage[russian,english]{babel} 
\usepackage{amsmath,amssymb}        
\usepackage{listings}
\usepackage{cmll}
\usepackage[left=15mm, top=1cm, right=15mm, bottom=1cm, nohead, footskip=1cm]{geometry}
\usepackage{fontspec} 
\usepackage{misccorr} 
\setmainfont{Times New Roman}
 
\newcommand{\problem}[1]{\textbf{Problem #1} \newline}
  
\newcommand{\solution}{\textbf{Solution} \newline}
 
\newcommand{\question}[2]{
	\begin{center}
		\Large{\textbf{#1 (#2)}} 
	\end{center}
}

\newcommand{\statement}[2]{\large{\textbf{#1}}}

 
\begin{document}

\begin{center}

\Huge Algorithms CW Conspect

\end{center}


\textbf{NC} - схемы полиномиального размера и полилогарифмической глубиной

\textbf{AC} - схемы, где на каждый вход может поступать полином битов. 

\begin{question}{Task}{1}

\begin{itemize}
             
	\item $Space(f^2) = Npspace(f)$

	\item Верно, доказывали в ДЗ
	
	\item Наоборот, если $PH = PSPACE$, то сигмы коллапсируются
	
	\item Верно

\end{itemize}
 
\end{question}

$L \subset NL \subset P \subset RP \subset NP \subset \Sigma^p_2 \subset \prod^p_3 \subset PH \subset PSPACE \subset NPSPACE \subset EXP$


\end{document}