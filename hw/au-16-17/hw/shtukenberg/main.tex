\documentclass{amsart}
\usepackage{ifxetex}
\ifxetex
  \usepackage{fontspec}
  \usepackage{xunicode}
  \usepackage{xltxtra}
  \usepackage{xecyr}
  \setmainfont[Mapping=tex-text,Ligatures=TeX]{CMU Serif}
  \usepackage{polyglossia}
  \setdefaultlanguage{russian}
\else
  \usepackage[utf8]{inputenc}
  \usepackage[T2A]{fontenc}
  \usepackage[english,russian]{babel}
  \usepackage{concrete}
\fi
\usepackage{amsthm,amsmath,amsfonts,amssymb}
\usepackage{fullpage}
\usepackage{eufrak}
\usepackage{listings}
\usepackage{color}
\usepackage{xcolor}

%\newtheorem{problem}{Задача}
%\newtheorem{solution}{Решение}

\begin{document}

  \definecolor{dkgreen}{rgb}{0,0.6,0}
  \definecolor{gray}{rgb}{0.5,0.5,0.5}
  \definecolor{mauve}{rgb}{0.58,0,0.82}  

  \newcommand{\problemset}[1]{
    
    \begin{center}
      \Large #1
    \end{center}
  }

  \newcommand{\problem}[1]{\textbf{Problem #1} \newline}
  
  \newcommand{\solution}{\textbf{Solution} \newline}
  
  %\newcommand{\sumn}{#1}{$\sum\limits_0^\infty{#1}$}

  \lstset{ %
    language=C++,                % the language of the code
    basicstyle=\footnotesize,           % the size of the fonts that are used for the code
    numbers=left,                   % where to put the line-numbers
    numberstyle=\tiny\color{gray},  % the style that is used for the line-numbers
    stepnumber=1,                   % the step between two line-numbers. If it's 1, each line 
                                    % will be numbered
    numbersep=5pt,                  % how far the line-numbers are from the code
    backgroundcolor=\color{white},      % choose the background color. You must add \usepackage{color}
    showspaces=false,               % show spaces adding particular underscores
    showstringspaces=false,         % underline spaces within strings
    showtabs=true,                 % show tabs within strings adding particular underscores
    frame=single,                   % adds a frame around the code
    rulecolor=\color{black!10},        % if not set, the frame-color may be changed on line-breaks within not-black text (e.g. comments (green here))
    tabsize=2,                      % sets default tabsize to 2 spaces
    captionpos=b,                   % sets the caption-position to bottom
    breaklines=true,                % sets automatic line breaking
    breakatwhitespace=false,        % sets if automatic breaks should only happen at whitespace
    title=\lstname,                   % show the filename of files included with \lstinputlisting;
                                    % also try caption instead of title
    keywordstyle=\color{blue},          % keyword style
    commentstyle=\color{dkgreen},       % comment style
    stringstyle=\color{mauve},        % string literal style
    escapeinside={\%*}{*)},            % if you want to add LaTeX within your code
    morekeywords={done, to},              % if you want to add more keywords to the set
  %  deletekeywords={...}              % if you want to delete keywords from the given language
  }

  \maketitle
\vspace{0.7cm}
  \problemset{Additional homework}

\textit{Исправлено: № 3, № 4.}

~\

~\

~\

\begin{problem}{1}

\end{problem}

~\

~\

\begin{solution}

Если L - полный язык для PH, то $L \in \Sigma_i^p$ для некоторого i.

Возьмем язык $L' \in PH$, он сводится к L, значит, $L' \in \Sigma_i^p$

\textit{Пояснение}

Что значит сводится? Есть полиномиальное сведение $x -> p(x)$

То есть для $L'$ верно

$x \in L' \Leftrightarrow \exists x_1 \forall x_2 ... x_i : M'(p(x), x_1, ..., x_i) = 1$

\end{solution}

~\

~\

\begin{problem}{2}


\end{problem}

~\

~\

\begin{solution}

$BP \cdot NP = \{L : L \leqslant_R 3SAT\}$

Во-первых, $3SAT \in NP \Rightarrow x \in 3SAT \Leftrightarrow \exists u : M_{3SAT}(x, u) = 1$

Во-вторых, $BPP \in \Sigma_2^p$, то есть для языка $B \in BPP$ верно $x \in B \Leftrightarrow \exists u \forall v M_B(x, u, v) = 1$

Мы рассматриваем множеств языков, сводимых к 3-САТУ некоторым алгоритмом B из BPP.

$M_B$ преобразует вход для 3-САТА и запускает его.

Таким образом $L \in BP \cdot NP \Rightarrow \exists u \forall v \exists u' M_{3SAT}(x, u, v, u') = 1$, то есть $L \in \Sigma_3^p$, то есть $BP \cdot NP \subset \Sigma_3^p$

\end{solution}

~\

~\


\begin{problem}{4}


\end{problem}

~\

~\

\begin{solution}

Пусть L разрешим, тогда сушествует машина M, которая его решает.

Попробуем решить HALT.

Заведем машину Тьюринга A.

Пусть $A_{x \: i}(t) = 0$, если $M_i$ на входе x работает более, чем за время t.

Иначе заставим A работать $100|t|^2 + 201$ шагов

Таким образом вот алгоритм решения HALT.

Принимаем на вход (i, x), i - номер машины, x - вход.

Запускаем $M(A_{x \: i})$. Если $A_{x \: i}$ останавливается до $100n^2 + 200$ шагов на всех входах, значит $M_i$ не останавливается на иксе никогда. Если все же существует t, такое, что $M_i$ останавливается за время t, то мы об этом узнаем, потому что найдется вход $A_{x \: i}$, на котором она будет работать долго.

Получается, мы решили HALT, противоречие

\end{solution}

~\

~\

\begin{problem}{5}

\end{problem}



\begin{solution}

Докажем по индукции, что $(\Sigma_i^p)^A = P^A$

База ясна: $(\Sigma_0^p)^A = P^A = (\Pi_0^p)^A$

Пусть $(\Sigma_i^p)^A = P^A = (\Pi_i^p)^A$ по предположению индукции

Докажем, что $(\Sigma_{i + 1}^p)^A = P^A$

$L \in \Sigma_{i + 1}^p \Leftrightarrow$

$x \in L \Leftrightarrow \exists u \in P(x) : M(x, u) = 1 $, где $M \in \Pi_i^p$

Значит,

$L \in (\Sigma_{i + 1}^p)^A \Leftrightarrow$

$x \in L \Leftrightarrow \exists u \in P^A(x) : M(x, u) = 1 $, где $M \in (\Pi_i^p)^A = P^A$, то есть $(\Sigma_{i + 1}^p)^A = (NP)^A = P^A$

$PH = \bigcup{\Sigma_i^p} \Rightarrow PH^A = P^A$


\end{solution}







\end{document}
