\documentclass{article}

\usepackage[T2A]{fontenc}           
\usepackage[russian,english]{babel} 
\usepackage[utf8]{inputenc}         
\usepackage{amsmath,amssymb}        
\usepackage{listings}
\usepackage{cmll}
\usepackage[left=15mm, top=1cm, right=15mm, bottom=1cm, nohead, footskip=1cm]{geometry}
\usepackage{fontspec} 
\usepackage{misccorr} 
\setmainfont{Times New Roman} 
\begin{document}

\footnotesize

\section{№ 1 Функции от многих переменных}

\subsection{ \footnotesize  № 1}

\subsection{ \footnotesize дарова влад № 2}

\subsection{ \footnotesize № 3 Обратные отображения. Оценка на норму обратного отображения. Теорема об обратимости отображений, близких к обратимым.}

Из линейной алгебры:

$A$ - обратим $\Leftrightarrow$

$A(R^n) = R^n \Leftrightarrow$

$\det A \neq 0 \Leftrightarrow$

$\forall x Ax = 0 \Rightarrow x = 0$

Оценка:

$||Ax|| \geq m||x|| \Rightarrow ||A^{-1}|| \leq \dfrac{1}{m}$

Понятно, что А обратим из п.4

$||A^{-1}y|| = ||A^{-1}(Ax) \leq \dfrac{||y||}{m}||$

...

Теорема: A - Обратимый оператор, $||A|| = \dfrac{1}{\alpha}$

B - оператор, $||B - A|| = \beta < \alpha$

Тогда 1. B - обратим 2. $||B^{-1} - A^{-1}|| \leq \dfrac{\beta}{\alpha(\alpha - \beta)}$

3. Множество обратимых линейных операторов открыто и отображение $A \rightarrow A^{-1}$ непрерывно

Пункт 1: нужно показать, что $||Bx|| \geq (a - b)||x|| > 0$

2. $B^{-1} - A^{-1} = B^{-1}(A - B)A^{-1}$

3. Открытость - это первый пункт

Непрерывность - это второй пункт, надо показать, что $A_n \rightarrow A \Rightarrow A_n^{-1} \rightarrow A^{-1}$, надо оценить $||A_n^{-1} - A^{-1}||$

\subsection{ \footnotesize № 6 Теорема о неявной функции}

\subsection{ \footnotesize № 7 Условный экстремум}

Если для любого x, который удовлетворяет условию $\Phi (x) = 0 f(a) \geq f(x)$ в окрестности

\section{№ 2 Теория меры}

\subsection{ \footnotesize № 10 Алгебра и $\sigma$-алгебра множеств..}

Определение:

X - большое множество, $A \in 2^X$, 

$\sigma_0, \delta_0, \sigma, \delta$

$A$ - симметрично $\Rightarrow (\delta_{(0)} \Leftrightarrow \sigma_{(0)})$

Алгебра множеств: Пустое принадлежит $A$, $A$ симметрично и обладает свойствами $\sigma_0, \delta_0$

$sigma-$алгебра - то же, но обладает сигмой	и дельтой.

Свойства алгебры множеств: 1. $\emptyset, X \in A$

2. $A \setminus B \in \textit A$

3. Пересечение и объединение конечного числа множеств также принадлежат алгебре

Утверждение: пересечение алгебр - алгебра

Руками берем и показываем

Теорема: $\varepsilon$ - система множеств, тогда существует единственная наименьшая по включению $\sigma$-алгебра, содержащая эпсилон

Попросту возьмем пересечение всех, которые содержат эпсилон

Борелевская оболочка системы - это как раз и есть такая наименьшая алгебра ее содержащая

Борелевская $\sigma$-алгебра - Борелевская оболочка все открытых множеств в $R^n$

\subsection{ \footnotesize № 11 Лемма про дизъюнктное объединение множеств. Кольцо и полукольцо. Теорема о свойствах элементов полукольца}

Лемма:

$\bigcup\limits^?{A_k} = \bigsqcup\limits^?{A_k \setminus (\bigcup\limits^{k - 1}{A_j})}$

Кольцо: для любых двух множеств из кольца верно, что там содержатся их объединение, пересечение и разность

$P$ - полукольцо, если

1. $\emptyset in P$

2. $\forall A, B \in P \Rightarrow A \cap B \in P$

3. Для любых A, B существует конечное число дизъюнктных множеств, таких, что разность A и B - это их дизъюнктное объединение

Теорема:

$P_i, P \in \textit{P} \Rightarrow \exists Q_1, ..., Q_m \in \textit{P} : P \setminus \bigcap{P_k} = \bigsqcup{Q_k}$

$P_i \in \textit{P} \Rightarrow \exists Q_{jk} : \bigcup{P_k} = \bigsqcup{\bigsqcup{Q_{jk}}}$


\subsection{ \footnotesize № 12 Произведение полуколец. Параллепипеды и ячейки. Связь между ними}

Декартово произведение полуколец - это множество всех декартовых произведений множеств колец.

Надо понять, что это полукольцо, порисовать прямоугольники

Определение: $\bigotimes [a_i, b_i]$ - параллелепипед $[a, b], a = [a_1...a_n], b = [b_1...b_n]$

Открытый параллелепипед - то же, но это не отрезки, а интервалы

Ячейка: полуинтервалы: слева включительно

Теорема: Любая ячейка представима в виде объединения счетного множества замкнутых параллелепипедов и пересечения счетного множества замкнутых

Открытые: $\bigotimes{[a_i - 1/k, b_i]}$

Замкнутые: $\bigotimes{[a_i, b_i - 1/k]}$


\subsection{ \footnotesize № 13 Полукольца ячеек. Представление открытого множества в виде объединения ячеек. Следствие}

Теорема: $P^n, P_\mathbb{Q}^n$ - полукольца, так как представимы в виде декартова произведения, по отдельности это тоже полукольца.

Теорема: всякое непустое открытое множество $G \in \mathbb{R}^n$ - объединение счетного числа дизъюнктных ячеек, замыкание которых содержится в G

Возьмем точку, она содержится с шариком. в шарике содержится какой-то кубик с рациональными координатами, выколем, что надо, это и есть ячейка, которая покрывает x, возьмем объединение таких ячеек по всем x из G, но различных кубиков счетное число, так как мы выбрали у него рациональные координаты, значит, получили счетное объединение, его можно превратить в счетное дизъюнктное объединение

Следствие: $B(P^n) = B(P_\mathbb{Q}^n) = B^n$ - Борелевская сигма-алгебра в $R^n$

1. $B(P^n) \supset B(P_\mathbb{Q}^n)$

2. $B^n \subset B(P_\mathbb{Q}^n)$

3. $B^n \supset B(P^n)$, любая ячейка лежит в $B^n$, так как является счетным объединением открытых множеств

\subsection{ \footnotesize № 14 Аддитивные функции множеств. Объем. Примеры. Свойства объема на полукольце}

$\varepsilon$ - система подмножеств X.

$\phi : \varepsilon \rightarrow (-\infty, +\infty]$

$\phi$ - аддитивная функция, если $\forall A, B \in \varepsilon : A \sqcup B \in \varepsilon$

$\phi(A \sqcup B) = \phi(A) + \phi(B)$

$\phi$ - конечно аддитивная функция, если $\forall A_1, .. , A_n \in \varepsilon : \bigsqcup{A_i} \in \varepsilon : \phi(\bigsqcup) = \sum$

Объем - конечно аддитивная функция $\phi$, заданная на полукольце подмножест X, причем достигает нуля только на пустом множестве и неотрицательна.

Свойства объема:

1. $P \subset Q \Rightarrow \mu P \leq \mu Q$

2. $P_i \subset Q$, $P_i$ - дизъюнктны, тогда $\sum{\mu P_k} \leq \mu Q$

В этих пунктах пользуемся леммой про полукольца, где разность раскладывается в дизъюнктное объединение

Второй пункт также верен и для бесконечности, можно просто перейти к пределу последовательности

3. $Q \subset \bigcup{P_k} \Rightarrow \mu Q \leq \sum{\mu P_k}$ - полуаддитивность

Опять-таки пользуемся какой-то леммой, представляем Q как объединение $P_k \cap Q$

\subsection{ \footnotesize № 15 Произведение объемов. Следствие для классического объема}

\subsection{ \footnotesize № 16 Мера: определение и примеры. Теорема о счетной полуаддитивности}

Мера - счетно-аддитивный объем

Теорема:

$\mu$ - мера $\Leftrightarrow \forall P_i : P \subset \bigcup P_i \Rightarrow \mu P \leq \sum$

Влево: $\mu$ - объем, значит нужно проверить счетную аддитивность, возьмем P, равный дизъюнктному объединению, для него мы знаем по свойству объема $\mu P \geq \sum$, получили два неравенства в разные стороны

Вправо: как обычно Возьмем $P_i' = P_i \cap P$, значит, P равен объединению P'-ов, запишем разложение объединения на диз объединение диз объединений. Получим: !$\mu P = \sum{\mu Q_{nk}} \leq \sum{\mu P_i'} \leq \sum{\mu P_i}$, чтд, первое равенство потому что мера, второе потому что Q дизъюнктны и содержатся внутри P-шек (важно, что их объединение как раз таки мб не равно P-шке)

Следствие:

P - сигма-алгебра, мю - мера

Тогда счетное объединение множеств меры 0 - множество меры 0, просто по теореме

\subsection{ \footnotesize № 17 Две теоремы о том, когда конечно аддитивная функция является мерой}

В начале этих теорем полагаем $\mu$ - объем, то есть надо проверить только счетную аддитивность

Теорема: $\mu$ - мера $\Leftrightarrow \forall A_i : A_{i} \subset A_{i + 1} \lim{\mu A_n} = \mu \bigcup{A_i}$ 

Вправо: Возьмем $B_i$ разности соседних, тогда A - это диз. объединение B. 
Так как мю - мера, то мера A - это сумма мер  B-шек, а сумма В-шек - это предел частичных сумм. При этом частичная сумма - это и есть $A_i$

Влево: опять таки для частичных суммы мы знаем аддитивность, потому что мю - это объем, еще мы знаем, что $A_n \rightarrow A$, а частичные суммы стремятся к счетной сумме. Значит напишем два предела и получим счетную аддитивность

Теорема: Живем в сигма алгебре, $\mu X \leq +\infty$

Тогда равносильно:

1. мю - мера

2. (непрерывность сверху) - i+1 ашка лежит в итой. мера пересечения - это предел $\mu A_n$

3. 2-е: пересечение пусто, значит мера равна 0

$1 \Rightarrow 2$

Поймем, что тут непрерывность снизу для дополнений. Тут важна конечность объема Х.

$2 \Rightarrow 3$ - очевидно

$3 \Rightarrow 1$

Возьмем дизъюнктное объединение С-шек, равным A, $A_n = \bigsqcup\limits_{n + 1}^\infty{C_k}$, тогда $A = \bigsqcup{C_i} \sqcup A_n$, там объединение первых n, берем предел, получаем счетную аддитивность

Следствие: если для какой-то Ашки верно, что ее мера меньше +беск, то можно принять ее равной X и 2е утв будет выполняться

\subsection{ \footnotesize № 18 Субмеры. Мера как сужение субмеры. Полная мера}

Субмера:

$\nu : 2^X \rightarrow [0, +\infty]$ - субмера, если

на пустом 0, монотонность, счетная полуаддитивность: $A \subset \bigcup{A_i}$, Тогда субмера A меньше/равна суммы

$\mu$ - полная мера, если для любого множества А, мера которого равна 0, любое ее подмножество в полукольце и мера его подмножества равна 0.

$\nu$ - субмера. Множество E $\nu$-измеримо, если $\forall A \subset X \nu A = \nu (A \setminus E) + \nu (A \cap E)$

Семейство ню-измеримых множеств образуют сигма-алгебру. Ню, суженная на эту алгебру - это полная мера

Что мы делаем:

1. Берем пустое множество, показываем, что оно ню-измеримо

2. Берем объединение двух ню-измеримых ...

3. Показываем симметричность

4. Показываем, что счетное дизъюнктное объединение тоже ню-измеримо

расписываем равенство для частичной суммы, там фигурирует А, засовываем под объединение 

Показываем, что эта штука для частичных сумм больше/равна $\sum{\nu (A \cap E_k)} + \nu (A\setminus E)$

Дальше пользуемся счетной полуаддитивностью

5. А обычное объединение в полукольце представляется в виду дизъюнктного объединения

6. ню - это объем на нашей алгебре(это доказать просто, а потом можно использовать конечную аддитивность)

7. Счетная полуаддитивность + объем = мера

8. Полная мера? Видимо, доказываем, что подмножество принадлжеит алгебре потому что там сумма двух нулей

Монотонность

\subsection{ \footnotesize № 19 Внешняя мера. Теорема о продолжении меры с полукольца}

Полукольцо называется хорошим, если объединение его элементов дает все множество X

Внешняя мера - инфимум по все суммам $\sum{\mu B_i} : A \subset \bigcup$, B - элементы полукольца, разумеется. Внешняя мера опять-таки задана на полукольце

Теорема: $\mu^*$ - субмера, совпадающая с $\mu$ на P. 

1. Показываем неравенства в разные стороны. Пользуемся счетной полуаддитивностью меры

2. Показываем счетную полуаддитивность для мю со *. 

Так как мы берем инфимум, то существуют такие $B_{ik} : \bigcup \supset A, \sum{\mu B_{ik}} < \mu^*A_n + \dfrac{\epsilon}{2^n}$

Что такое продолжение меры.

$\mu_0$ задана на хорошем полукольце, $\mu_0^*$ - ее внешняя мера, тогда сужения $\mu_0^*$ на алгебре измеримых ею функций. 

Теорема: на P $\mu$ и $\mu_0$ совпадают

Надо доказать, что E измеримо, то есть

$\mu A \geq \mu (A \cap E) + \mu (A \setminus E)$

1. Пусть A из полукольца, но меры на полукольцах совпадают, то есть мера от А раскладывается в сумму, (надо разность заменить на диз объединение)

Дальше напишем это равенство для субмеры (так как они совпадают) и воспользуемся полуаддитивностью

2. А - произвольное. 

По определению инфимума. есть $P_i : \bigcup{P_i} \supset A, \sum{\mu_0P_k} < \mu_0^* A + \epsilon$

Пэшки из полукольца, значит, для них равен первый пункт

Расписываем: $\mu_0^*(A) + \epsilon > \sum{\mu_0 P_k}$, применяем прошлый пункт(но там еще сумма)


\subsection{ \footnotesize № 20 Теорема о внешней мере множества. Следствие о структуре измеримых множеств}

сигма-конечная мера: если $\exists \{P_i\} : \mu P_k < + \infty, X = \bigcup{P_k}$

Теорема $\mu^* A < +\infty \Rightarrow \exists B_{n_k} \in P : C_n = \bigcup{B_{n_k}}, C = \bigcap{C_n} \supset A : \mu^* A = \mu C$

Возьмем такой набор бэшек: $C_n = \bigcup{B_{n_k}} \supset A$

$\mu C_n \leq \sum{\mu B_{n_k}} < \mu^* A + 1/n$

Инфимум, все дела.

Берем частичное объединение Сэшек, оно содержит А. Значит его мера больше/равна внешней меры А, но меньше  меры последнего С. Можно перейти к пределу, потому что там непрерывность меры сверху

Следствие:

\subsection{ \footnotesize № 21 Монотонный класс. Теорема о единственности продолжения меры}

Монотонный класс - система множеств. 

Для любой последовательности вложенных множеств их пересечение/объединение содержится в классе. В зависимости от того, в какую сторону включение

Теорема:

$\mu E = \nu E$ на полукольце, $\nu$ - мера на алгебре

Тогда если мю сигма-конечна, то на алгебре они тоже равны

1. Доказываем, что $\nu A \leq \mu A$.

Он равны на всех пэшках из инфимума, значит, в силу счетной полуаддитивности все выражения под inf больше/равны $\nu A$

2. Докажем, что $\mu(P \cap E) = \nu (P \cap E), E \in A$

воспользуемся определением субмеры и получим(надо расписать $\nu P = \mu P$)

3. $mu$ сигма-конечна, значит, существует разбиение X на Р из полукольца.

Дальше напишем $X \cap E, E \in A$ и все получим

\subsection{ \footnotesize № 22 Счетная аддитивность классического объема. Определение меры Лебега. Свойства 1-6}

Теорема: классический объем - сигма-конечная мера.

конечность ясна: можно разбить на единичные ячейки

Докажем полуаддитивность:

Возьмем $P \in P^m$

ячейку, для которой надо доказать. Она принадлежит неком объединению Пэшек. В объединении будем брать открытые параллелепипеды чуть больше, чем сами ячейки: на $\epsilon / 2^n$

Возьмем параллелепипед чуть меньше нашего исходной ячейки, выберем для него конечное открытое подпокрытие, для него мы знаем полуаддитивность, получим неравенство с эпсилоном

Устремим эпсилон к 0.

Затема устремим нашу ячейку для которой мы доказали к исходной

Определение меры Лебега:

Стандартное продолжение классического объема на $\mathbb{R}^n$

Свойства 1-6.

1. Открытые множества измеримы и имеют положительную меру

Так как борелевская алгебра включает все открытые множества

2. Все замкнутые множества измеримы, в силу симметричности

3. Всякое не несчетное множество имееет меру 0.

Точка - это предел отрезка. У меры вообще есть свойство: счетная аддитивность

4. Любое ограниченное измеримое множество имеет конечную меру, т.к содержится в какой-то ячейке

5. Любое измеримое множество - не более, чем счетное диз объединение измеримых. Наша мера сигма-конечная, то есть $\mathbb{R}^n$ можно разбить на счетное число ячеек, а потом каждую пересечь с A.

6. Если для любого $\varepsilon$ Найдутся множества A и B, $\lambda_m(B_\varepsilon \setminus A_\varepsilon) < \varepsilon \Rightarrow$ E - измеримо


\subsection{ \footnotesize № 23 Свойства 7-14. Пример несчетного множества, имеющего нулевую меру. Пример неизмеримого множества}

7. Если для любого эпс существует B, такой что Е содержится в В и при этом мера Bшек меньше эпс, тогда E измеримо, и его мера равна 0.

8. Счетное объединение множеств меры 0 дает множество меры 0

9. Если мера Е равна 0, то внутренность Е равна пустому множеству. Пусть не так, тогда Есть шарик, в нем есть ненулевая ячейка

10. $\lambda_m E = \inf{\{\sum{\lambda_m P_n} : P_n \in P^m_{(\mathbb{Q})}, E \subset \bigcup{P_n}\}}$

11. Если мера Е равна 0, то существуют кубические ячейки, покрывающие Е

Воспользуемся предыдущим свойством, нарежем на кубики

12. Мера гиперплоскости равна 0

Можно выбрать счетное покрытие ячейками, каждая имеет меру 0

13. Множество, содержащееся в счетном объединении координатных гиперплоскостей, тоже измеримо, и его мера равна 0.

14. мера параллепипеда(откр или закр) равна мере ячейки

Множества отличаются одной точкой, которая равна по мере 0: m = 1

$m \geq 2$ Добавляем/выкидываем нечто, что содержится в конечном объединении координатных плоскостей.

Примеры:

Несчетные: гиперплоскость

если m = 1, то берем отрезок бьем на три части и рекурсивно убираем середину

Неизмеримые:

Эквивалентность: разность чисел лежит в Q. Выберем одного представителя от 0 до 1 из каждого класса.


\subsection{ \footnotesize № 24 Регулярность меры Лебега. Следствия}

Теорема: E - измеримое множество, $E \subset \mathbb{R}^m$.

Тогда существует открытое множество G : $G \supset E, \lambda_m(G \setminus E) < \epsilon$

Мера Е задается через инфимум. Возьмем такое покрытие, что оно ближе к мере, чем на эпсилон.

Это покрытие превращаем в покрытие откр. параллелепипедами привычным методом и в итоге получаем $\sum{\lambda_m(a_n^\prime, b_n)} < \lambda_m E + 2\epsilon$

Назовем G объединением таких паралл., G открыто, также пользуемся счетной полуаддитивностью

Получаем, что разность мер G и E меньше 2 эпс.

Если же у Е беск. мера, то она представима в виде счетного объединения конечно-изм.

ДЛя каждого найдем $G_k - E_k < \epsilon/2^k$

Следствия:

1. Е измеримо, эпс больше 0. Существует множество $F \subset E : \lambda_m(E \setminus F) < \epsilon$

Берем из теоремы G для $R^m \setminus E$

2. Е измеримо, значит,

$\lambda E = \inf{\{\lambda G : E \subset G=open\}}$

$\lambda E = \sup{\{\lambda G : E \supset G=closed\}}$
	
$\lambda E = \sup{\{\lambda K : E \supset K = compact\}}$

вопрос в том, можно ли подойти сколь угодно близко. Можно из теоремы.

3. E - измеримо $\Rightarrow \exists K_i \subset K_{i + 1} .... \subset E$ - компакты и множество меры 0, которое в объединении с компактами дает Е.

Ну возьмем посл компактов, отличающихся на одну n-ную, перейдем к пределу

\subsection{ \footnotesize № 25 Инвариантность меры Лебега относительно сдвига. Относительно движения. Единственность такой меры}

Теорема. Измеримость и величина меры не меняется при сдвиге множества. $\mu E = \lambda (E + v)$

Меры на ячейках совпадают, значит, все в мире совпадает

Теорема. Пусть мера инвариантна относительно сдвига и мера каждой ячейки конечна. Тогда $\mu = k\lambda$

$k \in [0, +\infty)$

Примем k равным мере на единичной ячейке

Если k равно единице, то покажем про рациональные числа. берем ячейку со стороной $1/n$, все инвариантно относительно сдвига, короче надо просто перемножить, там все норм.

Если больше 0, то рассматриваем вспомогательную меру.

Если равно нулю, то мера на всем пространстве равна 0, так как это счетное объединение таких ячеек

инвариантность относительно движения

?

\section{Измеримые функции. Интеграл Лебега}

\subsection{ \footnotesize № 26 Измеримые функции. Примеры. Эквивалентные определения. Свойства}

Свойства измеримых функций

$f : E \rightarrow \overline{R}$ - измерима

1. $E\{a <(\leq) f <(leq) b)\}$ - измеримо. Пересечение измеримых

2. $E\{f = a\}$ измеримо

3. $|f|, -f$ измеримы

4. min, max тоже

5. Прообраз любого открытого множества измерим. Открытое множество представляется в виде беск объединения ячеек. А про ячейку мы все знаем, ну объединение можно вынести

6. $E = \bigcup E_i$, f суженное на $E_i$ измеримо для любого i, тогда f измеримо

Попросту $E\{f < a\} = \bigcup{E_n{\{f < a\}}}$

7. Каждая измеримая на E функция - сужение функции, измеримой на X

Покажем эту функцию: 0, если x не принадлежит E. Воспользуемся предыдущим пунктом

\subsection{ \footnotesize № 27 Измеримость инфимума, супремума, предела(верхнего, нижнего), композиции}

Инфимум и супремум представляются в виде бесконечного пересечения.

Верхний и нижний пределы выражаются через инфимумы и супремумы.

Обычный предел если есть, то равен верхнему.

?? композиция

\subsection{ \footnotesize № 28 Арифметические операции с измеримыми функциями. Измеримость непрерывной функции}

Теорема: 

1. Сумма, произведение измеримы

Если не бесконечности, то применяем теорему о композиции. Для бесконечностей там объединение пересечений изм. множеств

2. f измерима, $\phi$ - непрерывна, тогда $\phi \circ f$ измерима

3. $f \geq 0 \Rightarrow f^p$ измерима

Отдельно для бесконечностей

4. f измерима, тогда $\dfrac{1}{f}$ измерима на $E\{f \neq 0\}$

Следствие: Произведение конечного числа измеримых - измеримое

Линейная комбинация тоже

И натуральная степень тоже

Теорема: Непрерывная на E функция измерима

Просто говорим, что прообраз открытого множества измерим.

\subsection{ \footnotesize № 29 Простые функции. Свойства. Приближение неотрицательной измеримой функции простыми. Следствия}

Простая функция - измеримая функция, множество значений которой конечно

Допустимое разбиение X - разбиение на такие измеримые множества, что на каждом из них функция постоянна

Свойства

1. Если f, g - простые функции, то у них существует общее допустимое разбиение

$f = \sum\limits^n{c_k\mathbb{1}_{A_k}(x)}$

Получается для двух функций допустимое разбиение - это дизъюнктное объединение

2. Сумма, произведение, разность простых - простая

3. Линейная комбинация простых - простая

4. Поточечный максимум и поточечный минимум - простые

Теорема:

f неотрицательная и измеримая. Тогда сущ. посл. $\phi_i \leq \phi_{i + 1}$

эти функции простые, и их предел равен f.

Тогда если f ограничена, то $\phi_n$ можно выбрать так, что $\phi_n \rightrightarrows f$

Надо брать такие множества

$E_k = X\{\dfrac{k - 1}{2^n} \leq f(x) < \dfrac{k}{2^n}\}$

$E_0 = X{f = 0}$

$E_{4n + 1} = X\{f(x) \geq 2^n\}$

$\phi_n(x) = \dfrac{k - 1}{2^n}, x \in E_k, k = 1,2...4^n$

И для других x соответственно 0 или $2^n$ Ну короче все получится


Следствия: f измерима (неотрицательная или нет)$\Leftrightarrow f$ - поточечный предел простых

Если неотрицательная, то надо сказать $f = f_+ + f_-$

Влево: Предел измеримых - измеримая

\subsection{ \footnotesize № 30 Различные виды сходимости последовательности функций. Единственность с точностью до множества меры 0 предельных функций}

Равномерная сходимость (сходимость супремума по x)

Поточечная сходиомсть

Сходиомть почти всюду: сходится везде, кроме множества меры 0

Сходимость по мере: $\mu X\{|f_n - f| > \epsilon\} \rightarrow 0$

Утверждение:

1. $f_n \rightarrow g, f_n \rightarrow f$ почти всюду, значит, f = g почти всюду. Просто по определению

2. $f_n \Rightarrow f, f_n \Rightarrow f$ по мере, значит, f = g почти всюду

Берем $\mu X\{f - g > \epsilon\}$, оцениваем через последовательность, понимаем, что она равна 0

Значит, мера множества, где f не равно g равна 0

\subsection{ \footnotesize № 31 Теорема Лебега о сходимостях. Примеры}

Теорема Лебега:

$\mu E < +\infty, f_n, f$ измеримы, действуют из E.

$f_n \rightarrow f$ почти всегда

Тогда есть сходимость по мере.

Переопределим посл. в плохих точках

Дальше пусть $f_n \searrow f$, тогда $E\{f_n - f >\epsilon\} \supset E\{f_{n + 1} - f >\epsilon\}$

Дальше применяем непрерывность меры сверху, пересечение этих фигней пусто.

Общий случай: Надо рассмотреть первый случай для супремумов

НО


Обратное неверно: можно брать характеристические функции множеств, которые по мере стремятся к 0, но поточечно не сходятся

то что мера конечна тоже важно: можно взять $f_n = \mathbb{1}_{[n, +\infty)}$

\subsection{ \footnotesize № 32 Теорема Рисса. Следствие. Теоремы Егорова, Фреше и Лузина}

Из сходящейся по мере последовательности можно выбрать подпоследовательность, которая сходится почти всюду

Следствие:

$f_n \leq g$ почти везде и $f_n \rightarrow	f \Rightarrow f \leq g$ почти повсюду

Теорема Фреше:

$f : R^m \rightarrow R$ - измеримая функция, тогда $\exists f_n \in C(R^n): f_n \rightarrow f$ почти всюду

Теорема Лузина:

$E\subset R^m, f : E \rightarrow R$ - измерима

$\epsilon > 0 \Rightarrow \exists e \in R^m : \lambda e < \epsilon, f$ суженная на $E \setminus e$ - непрерывная

Теорема Егорова:

$f_n, f : E \rightarrow R$ измеримы

$f_n \rightarrow f$ почти всюду, $\epsilon > 0$. Тогда $\exists e \subset E, \mu e < \epsilon, f_n \rightrightarrows f$ на $E \setminus e$

\subsection{ \footnotesize № 33 Интеграл от простой функции. Свойства. Определение интеграла Лебега. Элементарные свойства интеграла неотрицательной функции}

Лемма

$f \geq 0$, простая. $A_i, B_i$ - допустимые разбиения X

$a_i, b_i$ - значения на соответствующих множествах. Тогда $\sum{a_k\mu A_k} = \sum{b_j\mu B_j}$

Интеграл: $\sum{a_k \mu(A_k \cap E)}$

1. инт конст.

2. монотонность интеграла.

Определение: интеграл от неотр. функции: супремум по интегралам всех простых функций не больше нее

\subsection{ \footnotesize № 34 Теорема Беппо Леви}

$f \geq 0$- измерима.

$0 \leq f_n \leq f_{n + 1}$ - измеримы

$\lim{f_n} = f$

$\lim{\int{f_n d\mu}} = \int{f}$






\end{document}