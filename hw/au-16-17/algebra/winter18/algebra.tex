\documentclass{article}

\usepackage[T2A]{fontenc}           
\usepackage[russian,english]{babel} 
\usepackage[utf8]{inputenc}         
\usepackage{amsmath,amssymb}        
\usepackage{listings}
\usepackage{cmll}
\usepackage[left=15mm, top=1cm, right=15mm, bottom=1cm, nohead, footskip=1cm]{geometry}
\usepackage{fontspec} 
\usepackage{misccorr} 
\setmainfont{Times New Roman} 
\begin{document}

\subsection{Тензора}

\tiny

\subparagraph{\tiny № 1 Тензорное произведение. Разложимые тензора}

\begin{flushleft}

Линейное пространство всех полилинейных функций $V_1 \times .... \times V_n \rightarrow U$ - полилинейное пространство $Hom(V_1, ..., V_n, U)$.

$M$ - множество всех функций, $\{ \sum{a_{v_1, .., v_n}(v_1, .. v_2)} : a_{v_1, .., v_n} \in K\}$

Важно, что в K.

$M_0 = \langle (v_1, .., v_k^\prime + v_k^{\prime \prime}, v_n) - (v_1, .., v_k^\prime , v_n) - (v_1, .., v_k^{\prime \prime}, v_n), \alpha(v_1, .., v_n) - (v_1, .., \alpha v_k, v_n) \rangle$

Тензорное произведение $V_1 \otimes V_2, ..., \otimes V_n - M \setminus M_0$

Разложимый тензор - класс эквивалентности функции $(v_1, ..., v_n)$
Разложимые тензоры порождают все тензорное произведение - лемма (потому что это классы эквивалентности порождающих элементов)
Есди одно из пространств тривиально, то их тензорное произведение тоже (нолик как коэффициент)

\end{flushleft}

\subparagraph{\tiny № 2. Связь между полилинейными отображениями и линейными отображнениями тензорного произведения пространств}


\begin{flushleft}

$t : V_1 \times ... \times V_n \rightarrow \bigotimes{V_i}$

$(v_1, v_2, ..., v_n) \rightarrow v_1 \otimes v_2 \otimes ... \otimes v_n$

1) t - полилинейно
2) t - универсально, то есть $\forall s \in Hom(V_1, ..., V_n, U) \exists !f \in Hom(\bigotimes{V_i}, U): s = f \circ t$

Доказательство:
1) Полилинейность очевидна
2) $f_1 \circ t = f_2 \circ t$

$f_1 (v_1 \otimes v_2 ... \otimes v_n) = f_2 (v_1 \otimes v_2 ... \otimes v_n)$

$(v_1 \otimes v_2 ... \otimes v_n)$ - образуют все пространство, значит, $f_1 = f_2$

Существование:
$g : M \rightarrow U$
$g(v_1, v_2, .., v_n) = s(v_1, v_2, ..., v_n)$
Сузим эту фигню, получим f
Теперь осталось показать, что для двух элементов из одного класса эквивалентности g равно.

Но это очевидно в силу полилинейности, отсюда вывод, что можно определять гомоморфизм только на разложимых тензорах, а дальше он естественныи образом продлится

Следствия:
Существуют канонические изоморфизмы векторных пространств:

1. $Hom(V_1, ..., V_n, U) \cong Hom(V_1 \otimes V_2 ... \otimes V_n, U)$
Изоморфизм очевиден: любому элементу s сопоставим f по универсальному свойству
2. $Hom(V_1, ..., V_n, K) \cong (V_1 \otimes V_2 ... \otimes V_n)^*$
Первый пункт
3. $\dim{(V_1 \otimes V_2 ... \otimes V_n)} = \dim{V_1} \dim{V_2} ... \dim{V_n}$
Размерность тензорного произведения = двойственного = произведению
4. $e_1^{(i)}, e_2^{(i)}, ..., e_n^{(i)}$ - базис $V_i$
Тогда $e_{j_1}^{(1)}, e_{j_2}^{(2)}, ..., e_{j_n}^{(n)}$
Покажем, что система образующих
Так как любой разложимый тензор представим в данном базисе

\end{flushleft}

\subparagraph{\tiny № 3. Расширение скаляров. Комплексификация пространства}

\begin{flushleft}

\textbf{Комплексификация пространства}

V - векторное пространство над R, тогда можно задать изоморфизм $C \otimes V \rightarrow V^C$
$1 \otimes e \rightarrow e$
$i \otimes e \rightarrow ie$

\textbf{Расширение скаляров}
$L, K$ - поля, $K \subset L$
V - пространство над K, тогда $L \otimes V$ - векторное пространство над L
Продолжим эту структуру, но над L
Тогда $\alpha \in L, \alpha (\beta \otimes v) = (\alpha \beta) \otimes v$ для базисных векторов $\beta \in L, v \in V$

\end{flushleft}

\subparagraph{\tiny № 4. Тензорное произведение пространств. Канонические изоморфизмы}

\begin{flushleft}

Лемма о существовании канонических изоморфизмов
$1. V_1 \otimes V_2 \cong V_2 \otimes V_1$
$v_1 \otimes v_2 \rightarrow v_2 \otimes v_1$

$2. (V_1 \otimes V_2) \otimes V_3 \cong V_1 \otimes (V_2 \otimes V_3)$
$v_1 \otimes v_2 \otimes v_3 \rightarrow v_1 \otimes (v_2 \otimes v_3)$

3. $V_1^* \otimes ... \otimes V_n^* \cong (V_1 \otimes ... \otimes V_n)^*$
Изоморфизм $V_1^* \otimes ... \otimes V_n^* \leftrightarrow Hom(V_1, ..., V_n, K)$

$f_1 \otimes f_2 ... \otimes f_n \rightarrow ((v_1 \times ... \times v_n) \rightarrow f_1(v_1)...f_n(v_n))$

4. $U^* \otimes V \cong Hom(U, V)$
$f \otimes v \rightarrow (u \rightarrow f(u)v)$

\end{flushleft}


\subparagraph{\tiny № 5. Тензорная алгебра векторного пространства}

\begin{flushleft}

$T_p^q(V)$ - тензорное произведение p штук $V^*$ и q штук V

$V^{\otimes p}$ - тензорная степень V.

Произведение тензоров
$(f \otimes g) (v_1, ..., v_p, v_1^\prime, ..., v_{p^\prime}^\prime, f_1, ..., f_q, f_1^\prime, .. f_{q^\prime})$ - берем первые p векторов и q функций и применяем к ним f, ко всему остальному g и перемножаем.

Коммутативности нет, понятно почему.
Есть такие свойства:

$a, b \in K, f_1, f_2 \in T_p^q(V), g \in T_{p^\prime}^{q^\prime}(V)$

1. $(af_1 + bf_2) \otimes g = af_1 \otimes g + bf_2 \otimes g$
2. $f \otimes (ag_1 + bg_2) = af\otimes g_1 + bf \otimes g_2$
3. $(f \otimes g) \otimes h = f \otimes (g \otimes h)$

Тензорная алгебра:
$T(V) = \bigoplus\limits_{p, q}{T_p^q(V)}$, на ней определено умножение.
Аналогичная конструкция: многочлены.


\end{flushleft}

\subparagraph{\tiny № 6. Изменение координат тензора при замене базиса}

\begin{flushleft}

Есть обычное представление $T = \sum{T^{...}_{...}e^{i_1} \otimes e^{i_2} ... \otimes e_{j_1} \otimes ... \otimes e_{j_q}}$, $T_{...}^{...}$ - это невъебический коэффициент.

1. $e, f$ - базисы V, $f = eA, A = C_f^e$
2. $e^*, f^*$ - двойственные базисы $V^*$, матрица перехода равна $B = (A^T)^{-1}$ 

\end{flushleft}


\subparagraph{\tiny № 7. Симметрические тензоры. Симметрическая алгебра векторного пространства}

\begin{flushleft}

V - конечномерно, K - нулевой характеристики
$S_q$ действует на $T^q_0(V)$

$f_\sigma$ - такая функция, что $f(v_1 \otimes v_2, ... \otimes v_n) = v_{\sigma(1)} \otimes v_{\sigma(2)} .... \otimes v_{\sigma(n)}$

$S^q(V) = \{T \in T^q_0(V) | f_\sigma T = T \forall \sigma \in S_q$
$S = \dfrac{1}{q!} \sum\limits_{\sigma \in S_q}{f_\sigma}$

Утверждения: 
1. $S^2 = S$
2. $T_1 = f_\sigma T_2$, тогда $S(T_1) = S(T_2)$
3. $Im(S) = S^q(V)$

Лемма:
Пусть e - базис V, тогда 
1. $\{S(e_{i_1}, e_{i_2}, ..., e_{i_q}\}$ - базис $S^q(V)$
2. $S^q(V) \cong \{f \in K[x_1, ..., x_n] | \deg{f} = q\}$
3. $\dim{S^q(V)} = {n + q - 1 \choose q}$
$S(V) = \bigoplus{S^q(V)}$ - симметричная алгебра.

Введем умножение:
$T_1 \in S^q(V), T_2 \in S^p(V)$

$T_1 T_2 = S(T_1 \otimes T_2)$

Утверждение:
1. $S(V)$ - коммутативная ассоциативная алгебра

2. $(e_1^{k_1} ... e_n^{k_n})(e_1^{l_1} .... e_n^{l_n}) = e_1^{k_1 + l_1} ... e_n^{k_n + l_n}$

\end{flushleft}

\subparagraph{\tiny № 8. Внешняя алгебра или алгебра Грассмана}

\begin{flushleft}

Пространство антисимметричных тензоров: $\Lambda^q(V) = \{T \in T_0^q(V) | f_\sigma(T) = sign_\sigma T\}$
Оператор альтернирования:
$A = \dfrac{1}{q!} \sum\limits{\sigma in S_q}{sign_\sigma f_\sigma}$
Утверждения:
1. $T_1 = f_\sigma T_2 \Rightarrow AT_1 = sign_\sigma AT_1$
2. $A^2 = A$
3. $ImA = \Lambda^q(V)$
Утверждение:
$e_{i_1} \wedge e_{i_2} .... \wedge e_{i_n} = 0 \Rightarrow i_l = i_s$

Утверждение:
1. $q \leq n \Rightarrow e_{i_1}, e_{i_2}, ..., e_{i_q}$ - базис $\Lambda^q(V), 1 \leq i_1 \leq ... \leq i_q \leq n$

2. $q > n \Rightarrow \Lambda^q(V) = 0$

3. $\dim{\Lambda^q(V)} = {n \choose q}$

Следствие: 
$\dim{\bigoplus\limits_{q = 0}^n{\Lambda^q(V)}} = 2^n$
$T_1 \in \Lambda^p(V), T_2 \in \Lambda^q(V)$
$T_1 \wedge T_2 = A(T_1 \otimes T_2)$ - внешнее произведение

\end{flushleft}


\subparagraph{\tiny № 9. Вещественная структура на комплексном пространстве}

\begin{flushleft}

$V$ - векторное пространство над C.
Оператор комплексно антилинеен, если $\sigma (zv) = \overline{z} \sigma(v) \forall z \in C$
Утверждение:
$\sigma$ комплексно антилинеен: $V_R \rightarrow V_R$, $\sigma^2 = id$

$V = \mathbb{C} \otimes_R W, W = Ker(\sigma - id)$

\end{flushleft}

\subparagraph{\tiny № 10. Тело кватернионов}

\begin{flushleft}

Есть четырехмерное пространство $V = C^{2 \times 2}$
Комплексно антилинейный оператор $\sigma$:
$\sigma: A \rightarrow \overline{Adj(A)}^T$

$(a, b, c, d) \rightarrow (\overline{d}, -\overline{c}, -\overline{b}, \overline{a})$

Утверждение:
1. $\sigma(AB) = \sigma(A) \sigma(B), \forall A, B$ - матрицы
2. $\sigma(zA) = \overline{z} \sigma(A)$
3. $\sigma^2 = id$

Утверждение:
$V_R = \ker{(\sigma^2 - id)} = \ker{(\sigma - id)} \oplus \ker{(\sigma + id)}$

Тогда базис пространства: $1 = (1, 0, 0, 1), i = (i, 0, 0, -i), j = (0, 1, -1, 0), k = (0, i, i, 0)$

$\ker{(\sigma - id)}$ - тело кватернионов

1. $i^2 = j^2 = k^2 = ijk = -1$
2. $ij = -ji = k, jk = -kj = i, ki = -ik = j$
Утверждение:
$R \rightarrow H$, переходит в коэффициент при 1.

$I$ - подпространство чисто мнимых кватернионов
$I = \{A \in C^{2 \times 2} | \overline{A}^T = -A, tr(A) = 0 \}$
Сопряжение: минусики перед мнимыми частями.
Введем скалярное произведение $\langle u, v \rangle = Re(u * \overline{v})$
И норму: $||u||^2 = Re(u * \overline{u})$
Лемма.
1. Для любого ненулевого u - кватерниона существует обратный элемент $u^{-1} = \dfrac{\overline{u}}{||u||^2}$

2. Если $u, v \in I$, то $Im(uv) = u \times v$, где $\times$ - векторное произведение

\end{flushleft}

\subsection{\tiny Теория Галуа}

\subparagraph{\tiny № 11. Конечные и алгебраические расширения полей. Башня расширений}

\begin{flushleft}

Поле $L \supset K$ называется расширением K.
Степень расширения - размерность L как векторного пространства над K: $[L : K]$

Алгебраический элемент: $\alpha \in L : \exists 0 \neq f \in K[x] : f(\alpha) = 0$

Лемма:
1. Любое конечное расширение алгебраическое
2. $M \subset K \subset L \Rightarrow [L : M] = [L : K] * [K : M]$
1. Так как $\alpha^0, \alpha, ..., \alpha^n$ линейно зависимы.
2. $[K : M] = n, [L : K] = m$

u - базис K над M, v - базис L над K
Тогда $u_iv_j$ - базис L над M.

Система образующих очевидно, линейная независимость тоже.
Минимальный многочлен: такой $f \in K[x]$ со старшим коэффициентом 1, что идеал $I_\alpha$ многочленов, равных нулю в $\alpha$, равен $f(x)K[x]$

Минимальный многочлен существует для любого алгебраического элемента: пользуемся тем, что $K[x]$ - кольцо главных идеалов, и что он не пуст. Плюс домножаем на элементы из K, чтобы получить единицу

Свойства: минимальный многочлен неприводим, $I_\alpha$ максимален

Док-во: Ну понятно, если он приводим, то не может образовывать весь идеал. Любой простой идеал максимален.
Простой идеал - такой, что факторколько по нему - область целостности. 
Максимальный идеал - не содержится ни в каком другом.

\end{flushleft}


\subparagraph{\tiny № 12. Строение расширения $K(\alpha)/ K$}

\begin{flushleft}

$K(\alpha)$ - поле, являющееся пересечением всех таких M, что $\alpha \in m, K \subset M \subset L$
$K[\alpha]$ - множество значений всех многочленов из $K[x]$ в точке $\alpha$
$K[\alpha]$ - кольцо. $K[\alpha] \subset K(\alpha)$, так как любое поле содержащее ${\alpha, K}$ содержит и все значения многочленов
Утверждение:
1. $K \subset L, \alpha \in L$, тогда если $\alpha$ алгебраическое, то $K(\alpha) = K[x] / f_\alpha$
2. Если же оно трансцендентное, то $K(\alpha) \cong K(x) = Quot(K[x])$

Есть гомоморфизм колец $K[x] \rightarrow K[\alpha]$, $g(x) \rightarrow g(a)$

$K[\alpha] \cong K[x] / I_a$ по теореме о гомоморфизме

1. $\alpha$ - алгебраический, тогда $K[x] / I_\alpha$ - поле, которое содержит $\alpha, K$
2. $\alpha$ - трансцендентный, тогда изоморфны не только $K[x]$ и $K[\alpha]$, но и $Quot(K[x]) \cong Quot(K[\alpha])$, тогда показываем включения $Quot(K[\alpha])$ и $K(\alpha)$ в обе стороны

\end{flushleft}

\subparagraph{\tiny № 13. Присоединение корня неприводимого многочлена}

\begin{flushleft}

Если два расширения изоморфны, и изоморфизм тождественен на K, то она называются эквивалентными.
Утверждение:
Если K - поле, $f(x) \in K[x]$ - неприводим, тогда существует единственное с точностью до эквивалентности расширение вида $K(\alpha) : f(\alpha) = 0$

Док-во $I = fK[x], M = K[x] / I$, получается такое вложение $K \rightarrow M, a \rightarrow a + (f)$
В качестве $\alpha$ возьмем $\overline{x}$ - класс многочлена $x : x + qf$
Теперь альфа алгебраический, значит, можно сказать, что расширение $K(\alpha) \cong K[x] / f$, значит, они все будут эквивалентны

\end{flushleft}


\subparagraph{\tiny № 14. Конечные расширения полей. Поле разложения многочлена}

\begin{flushleft}

$K(a_1, a_2, ..., a_n)$ - минимальное поле, содержащее K и набор ашек.
Лемма:
1. Оно существует и единственно с точностью до эквивалентности
2. $K(a_1, ..., a_n) = K(a_1) (a_2)... (a_n)$

1. Без док-ва
2. Докажем по индукции, переход очевиден из определений

Следствие: не имеет значения в каком порядке добавлять элементы

Лемма:
1. Конечные расширения и только они получаются присоединением конечного числа алгебраических элементов.

В одну сторону потому что каждое из расширений в цепочке конечно, в другую надо сделать индукцию по степени расширения, пока она не 1 находим элемент, присоединяем к текущему полю

2. $a, b$ - алгебраические над K, тогда $a\pm b, a/b, ab$ - тоже алгебраические

$K(\alpha, \beta)$ - конечное расширения, и вся эта хрень в нем содержится(значит, они алгебраические)

3. Если $M \subset K, K \subset L$, то $M \subset L$ - алгебраическое расширение
Расммотрим $\alpha \in L$, и покажем, что он алгебраический над M.
Для этого, так как он алгебраический над K, можно представить $a_0 + ... + a_n \alpha^n = 0$, коэффициенты из K. Эти коэффициенты из линейной комбинации алгебраически над M, значит, сначала присоединим их, а тогда расширение $[M(a_1, ..., a_n, \alpha) : M]$ - конечное, а значит, $\alpha$ алгебраический над M.

Поле разложения многочлена - поле, полученное присоединением всех его корней.

\end{flushleft}

\subparagraph{\tiny № 15. Алгебраическое замыкание поля}

\begin{flushleft}

Поле K алгебраически замкнуто, если любой многочлен над K имеет корень.

Теорема. Для всякого K существует алгебраическое расширение $\overline{K}$, алгебраически замкнутое.

Док-во:
1. Построим расширение $L_1, K \subset L_1$, в котором любой неконстантный многочлен $f \in K[x]$ имеет корень в $L_1$.

Введем бесконечный набор переменных, каждая соответствует многочлену $x_f \leftrightarrow f \in K[x]$

$I = \langle f(x_f) | f \in K[x] \rangle$, он не совпадает со всем кольцом, потому что пусть содержится 1, тогда рассмотрим $K \subset F$ - получено присоединением всех корней многочленов f.

Равенство должно было остаться, но подставим теперь эти корни и отсосем.

Рассмотрим максимальный идеал $R \neq M \supset I$. $L_1 = R/M$
Есть естественные отображения $K \rightarrow R, R \rightarrow L_1$

Надо показать, что композиция этих отображений инъективна. Но тогда разность двух элементов из K лежит в M, а значит, там лежит 1, противоречие. 
Осталось понять, что любой неконстантный многочлен имеет корень в $L_1$.

2. Теперь построим бесконечное кол-во $K \subset L_1 \subset L_2 \subset ...$, возьмем в качестве L объединение их всех. При этом получается, что L алгебраически замкнуто.

3. Теперь построим $\overline{K}$
$K \subset M \subset L$. Требуемое расширение - объединение всех таких M, что M - алгебраическое расширение K. Покажем, что эта хрень алгебраически замкнута. Всякий многочлен $f \in \overline{K}[x]$ имеет корень в L, тогда этот корень алгебраический над $\overline{K}$, а значит $\overline{K}(\alpha)$ лежит в объединении.

\end{flushleft}

\subparagraph{\tiny № 16. Продолжение изоморфизмов. Сопряженные элементы}

\begin{flushleft}
$\sigma$ - гомоморфизм полей
$f(x) = a_0 + a_1x + ... + a_nx^n \in K[x]$
$f^\sigma(x) = \sigma(a_0) + ... + \sigma(a_n)x^n \in L[x]$

Пусть $K \subset L, K^\prime \subset L^\prime$
Тогда если есть гомоморфизм $\phi: K \rightarrow K^\prime$

То $\phi^\prime : L \rightarrow L^\prime$ - продолжение $\phi$, если на K они совпадают 

Утверждение:
$\phi: K \rightarrow K^\prime$ - изоморфизм.

f неприводим, $f(a) = 0, f^\phi(a^\prime) = 0$
1. Тогда $\phi$ продолжается до изоморфизма $\phi^\prime : K(\alpha) \rightarrow K^\prime(\alpha^\prime)$
При этом $\phi^\prime(\alpha) = \alpha^\prime$
Всякий элемент представим в виде $\sum\limits_0^{\deg{f} - 1}{a_i\alpha^i}$

Переведем его так $\sum\limits_0^{\deg{f} - 1}{a_i\alpha^i} \rightarrow \sum\limits_0^{\deg{f} - 1}{\phi(a_i)\alpha^{\prime i}}$

2. Тогда $\phi$ продолжается до изоморфизма полей разложения многочленов $f$ и $f^\sigma$
Последовательно. Возьмем корень f - $\alpha$. Тогда $\phi(\alpha)$ - корень $f^\phi$
Короче поля изоморфны на каждом шаге.

\textbf{Сопряженные элементы}
L - расширение K
Тогда группу автоморфизмов L, тождественных на K обозначим $Aut_KL$

Сопряженные элементы - такие $\alpha, \beta \in \overline{K}$, что $\alpha = \sigma \beta$, где сигма - это автоморфизм $\overline{K}$
Свойства:
1. Если $\alpha$ и $\beta$ сопряжены, то $f(\alpha) = 0 \Leftrightarrow f(\beta) = 0$
Тогда $f(\alpha) = f^\sigma(\alpha) = f^\sigma(\sigma \beta) = 0$, первый переход следует из того, что сигма тождественна на K, последний из предыдущей леммы.

2. Любые два корня неприводимого многочлена сопряжены.
Напрямую следует из предыдущего всего.
Следствие: множество сопряженных с $\alpha$ элементов совпадает с множеством корней $f_\alpha$

\subparagraph{\tiny № 17. Кратные корни многочлена}

$f$ - многочлен над K.
Рассматриваем корни f в алгебраическом замыкании
Тогда
1. У f есть кратный корень $\Leftrightarrow$ f и $f^\prime$ имеют общий корень
2. $f$ неприводим, $\deg{f} > 1$, тогда f имеет кратные корни тогда и только тогда, когда $f^\prime = 0$

Лемма:
$f$ неприводим, $\deg{f} > 1$
Тогда
1. Если $Char(K) = 0$, тогда f не имеет кратных корней
2. Если $Char(K) = p > 0$, то f имеет кратные корни тогда и только тогда, когда $f(x) = g(x^p)$

1. Пусть есть, но тогда f константа,  противоречие
2. Влево получается, что производная равна 0
Вправо берем производную, смотрим на коэффициенты кратные и некратные p.

$f(x) = g(x^{p^c})$, и g не имеет кратных корней


\end{flushleft}

\subparagraph{\tiny № 18. Редуцированная степень многочлена}

\begin{flushleft}

Тогда назовем $\deg{g}$ редуцированной степенью f.

Утверждение
$f = f_\alpha$ раскладывается на линейные множители в поле L.
$\sigma : K \rightarrow L$

Тогда существует ровно m продолжений $\sigma$ до вложения $\overline{\sigma} : K(\alpha) \rightarrow L$
Где m совпадает с редуцированной степенью f, если поле характеристики не 0, а если 0, то просто со степенью f.

Док-во:
Каждое продолжение гомоморфизма переводит $\alpha$ в сопряженный элемент, чем полностью и определяется. Тогда, если кратных корней нет(Char = 0), то все ясно, иначе, опять-таки переходим к $g(x^{p^c})$, в котором нет кратных корней

$f(x) = \prod{x^{p^c} - \beta_i}$, пусть $\alpha_i$ - корень f.
Ну а дальше понимаем, что $x^{p^c} - \beta_i = (x - \alpha_i)^{p^c}$
\end{flushleft}

\subparagraph{\tiny № 19. Сепарабельная степень расширения}

\begin{flushleft}

L - конечное расширение. M алгебраически замкнуто
Тогда если $\sigma : K \rightarrow M$ - вложение, то количество продолжений $\sigma$ до вложения $L \rightarrow M$ назовем сепарабельной степенью

Утверждение:
Пусть $K \subset L \subset M$
Тогда 1. $[M : K]_s = [M : L]_s [L : K]_s$
2. $[L : K]_s \leq [L : K]$
Док-во 1. Сначала каким-то количеством способов продолжаем L, потом M.

2. По очереди рассматриваем присоединенные элементы, получается проивзедение степеней.
В одном случае сепарабельных, в другом - расширения, но степень расширения - это $\deg{f}$, а сепарабельная - это редуцированная степень минимального многочлена

\end{flushleft}


\subparagraph{\tiny № 20. Конечные поля. Гомоморфизм Фробениуса}

Утверждение.
F - конечное поле, тогда $|F| = p^k$, p - простое, $p = Char(F)$, $a^{Char - 1} = 1, a \in F^*$
Лемма. Корни многочлена $f(x) = x^{p^n} - x$ образуют поле.
$\alpha, \beta$ - корни, тогда $\alpha^{p^n} = \alpha ... $
$a^{p^n} + b^{p^n} = (a + b)^{p^n} = (a + b)$ 

C минусом все норм.
$a^{p^n - 2} = a^{-1}$

Лемма. Конечные поля одинакового порядка изоморфны
Любой элемент поля K удовлетворяет $x^{p^k} - x = 0$
Короче, все элементы K являются корнями f. Короче будем расширять $F_p$ до K добавляя корни f, и в итоге все элементы будут корнями
Значит, K изоморфно полю разложения $f$, а все поля разложения над одним полем изоморфны

Утверждение
1. $\forall p \in P, \forall n \in N \exists$ поле из $p^n$ элементов
2. $\forall m$ существует одно с точностью до эквивалентности расширение поля $F_q$ степени m.

1. Рассмотрим поле разложения $x^{p^n} - x$, корни образуют поле, все они различны, так как кратные корни могут быть только у многочлена вида $g(x^p)$
2. $[K : F_q] = m$, а конечные поля одинакового порядка изоморфны.

Автоморфизм $\phi : F_q \rightarrow F_q, x \rightarrow x^p$ - это эндоморфизм Фробениуса
Утверждение
1. $\phi$ - автоморфизм $F_q$
2. Aut $F_q = \langle \phi \rangle$, |Aut $F_q$| = n.
То есть эта группа циклическая.

\end{document}