\begin{para}{Wow}

$k \leftarrow Gen(1^n)$

$((m_0, m_0'), (m_1, m_1')) \leftarrow Adv(1^n)$

$b \leftarrow \{0, 1\}$

$(c, c') = Enc_k(m_b, m_b')$

$b' \leftarrow Adv((c, c'), 1^n)$

$b = b'$

Вероятность того, что угадают, надо сравнивать с $\dfrac{1}{2} + negl(n)$

\end{para}


\begin{para}{Взлом на основе выбранных текстов}

Есть некий черный ящик, оракул - $Enc_k$ в котором зашит ключ. 

То, как он работает известно атакующему, неизвестен только ключ

$k \leftarrow Gen(1^n)$

$((m_0, m_1) \wedge |m_0| = |m_1| \leftarrow Adv(1^n, Enc_k)$

$b \leftarrow \{0, 1\}$

$c \leftarrow Enc_k(m_b)$

$b' \leftarrow Adv(c, 1^n, Enc_k)$

$b = b'$

Шифр является устойчивым к атакам на основе выбранных сообщений, если вероятность того, что угадают $\leq \dfrac{1}{2} + negl(n)$

\end{para}

\begin{para}{Случайно распределенная функция}

Возьмем функцию с ключом $F : \{0, 1\}^n \times \{0, 1\}^n \rightarrow \{0, 1\}^n$. Первый аргумент - это k, второй - x.

$k \leftarrow Gen(1^n)$

$F_k : \{0, 1\}^* \rightarrow \{0, 1\}^*$

Функция F случайно распределенная, если мы берем случайный ключ и получаем случайную функцию.
$F_k$ неотличима от случайной, если $|Pr[D^{(F_k, 1^n)} - Pr[D^{(f, 1^n)}]| \leq negl(n)$


\end{para}