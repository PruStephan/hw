\problemset{KEK}

\begin{para}{Экзистенциальные типы}

\begin{enumerate}

  \item $\Gamma \vdash \phi [\alpha = \theta] \Rightarrow \Gamma \vdash \exists \alpha . \phi$

  \item $\Gamma \vdash \exists \alpha . \phi, \Gamma, \phi \vdash \psi \Rightarrow \Gamma \vdash \psi$

\end{enumerate}

Что такое стек:

$\forall \nu \exists \tau ( \tau \& ((\tau \& \nu) \rightarrow \tau) \& (\tau \rightarrow (\tau \& \nu))$

Как стек связан с экзистенциальными типами: $\alpha \sim \tau$. Тау - это тип стека, ню - тип значения. Фи - это упорядоченная тройка. Пси - это код, который чего-то делает

pack (M, t) to $\exists \alpha . \sigma$. M - реализация, t - тип стека, альфа - имя стека, сигма - интерфейс.

pack (M, t) to $\exists \alpha . \sigma$ = $\wedge \beta . \lambda x^{\forall \alpha ( \sigma \rightarrow \beta)}. x t M$

Пример. class W implements Stack

$W = \tau, Stack = (\alpha, \sigma)$, M - тело класса

Отличие дженериков от экзистенциального типа : дженерик говорит дайте тип, и я это сделаю. Экз. тип говорит существует такая реализация?? что я это сделаю


\end{para}

\begin{para}{Типовая система Хиндли-Милнера}

\textbf{Система F неразрешима}

\begin{defe}{Type rank}

R(0) - все типы без кванторов

R(x + 1) = R(x0 | R(x) -> R(x + 1) | $\forall \alpha R(x + 1)$

\end{defe}

\end{para}