\documentclass{amsart}
\usepackage{ifxetex}
\ifxetex
  \usepackage{fontspec}
  \usepackage{xunicode}
  \usepackage{xltxtra}
  \usepackage{xecyr}
  \setmainfont[Mapping=tex-text,Ligatures=TeX]{CMU Serif}
  \usepackage{polyglossia}
  \setdefaultlanguage{russian}
\else
  \usepackage[utf8]{inputenc}
  \usepackage[T2A]{fontenc}
  \usepackage[english,russian]{babel}
  \usepackage{concrete}
\fi
\usepackage{amsthm,amsmath,amsfonts,amssymb}
\usepackage{fullpage}
%\usepackage{eufrak}
\usepackage{listings}
\usepackage{color}
%\usepackage{xcolor}

\newtheorem{problem}{Задача}

\newenvironment{para}[1]
{
\begin{Large}
\textbf{#1}
\newline
\end{Large}
}
{
\vspace{0.7cm}
}

\newenvironment{defe}[1]
{
\vspace{0.2cm}
\textbf{Def} \textit{#1}
}
{
\vspace{0.5cm}
}


\begin{document}

  \definecolor{dkgreen}{rgb}{0,0.6,0}
  \definecolor{gray}{rgb}{0.5,0.5,0.5}
  \definecolor{mauve}{rgb}{0.58,0,0.82}  

  \newcommand{\problemset}[1]{
    
    \begin{center}
      \Large #1
    \end{center}
  }

  \lstset{ %
    language=C++,                % the language of the code
    basicstyle=\footnotesize,           % the size of the fonts that are used for the code
    numbers=left,                   % where to put the line-numbers
    numberstyle=\tiny\color{gray},  % the style that is used for the line-numbers
    stepnumber=1,                   % the step between two line-numbers. If it's 1, each line 
                                    % will be numbered
    numbersep=5pt,                  % how far the line-numbers are from the code
    backgroundcolor=\color{white},      % choose the background color. You must add \usepackage{color}
    showspaces=false,               % show spaces adding particular underscores
    showstringspaces=false,         % underline spaces within strings
    showtabs=true,                 % show tabs within strings adding particular underscores
    frame=single,                   % adds a frame around the code
    rulecolor=\color{black!10},        % if not set, the frame-color may be changed on line-breaks within not-black text (e.g. comments (green here))
    tabsize=2,                      % sets default tabsize to 2 spaces
    captionpos=b,                   % sets the caption-position to bottom
    breaklines=true,                % sets automatic line breaking
    breakatwhitespace=false,        % sets if automatic breaks should only happen at whitespace
    title=\lstname,                   % show the filename of files included with \lstinputlisting;
                                    % also try caption instead of title
    keywordstyle=\color{blue},          % keyword style
    commentstyle=\color{dkgreen},       % comment style
    stringstyle=\color{mauve},        % string literal style
    escapeinside={\%*}{*)},            % if you want to add LaTeX within your code
    morekeywords={done, to},              % if you want to add more keywords to the set
  %  deletekeywords={...}              % if you want to delete keywords from the given language
  }

  \maketitle
\vspace{0.7cm}
  \problemset{KEK}

\begin{para}{Строим новый нахуй язык}

$\Phi = p \: \: | \: \: (\Phi \rightarrow \Phi) \: \: | \: \: \forall p.\Phi$

Простой случай $p \in \{0, 1\} \Rightarrow \forall p. p \rightarrow p = T$

$p \rightarrow Q$ - обычная импликация

$\forall p.Q$ - подставляем все значения в Q и берем конъюнкцию

Правила вывода

$\Gamma \vdash \phi \: \: | \: \: \Gamma \vdash \forall p.\phi, p \notin FV(\Gamma)$

$\Gamma \vdash \forall p.\phi \: \: | \: \: \Gamma \vdash \phi[p = \theta]$

Сокращения

\begin{itemize}

  \item $\perp = \forall a.a$ - ложь

Из этого следует, что из лжи выводится что угодно

  \item $\phi \wedge \psi = \forall a. ((\phi \rightarrow \psi \rightarrow a) \rightarrow a)$ - конъюнкция
  
  \item $\phi \vee \psi = \forall a. ((\phi \rightarrow a) \rightarrow (\phi \rightarrow a) \rightarrow a)$ - дизъюнкция
  
  \item $\exists p. \phi = \forall b. ( \forall p. (\phi \rightarrow b)) \rightarrow b$ - квантор существования

\end{itemize}

\end{para}



\begin{para}{F system}

Язык

$L = x \: \: | \: \: \lambda x : \tau. L \: \: | \: \: LL \: \: | \: \: \bigwedge \alpha. L \: \: | \: \: L \tau \: \: | \: \: $

$\bigwedge \alpha. L$ - типовая абстракция

$L \tau$ - типовое применение

Типы

$\tau  = \alpha \: \: | \: \: (\tau \rightarrow \tau) \: \: | \: \: ( \forall \alpha. \tau)$

Новые правила вывода

$\Gamma \vdash M : \sigma \Rightarrow \Gamma \vdash (\wedge \alpha. M) : \forall \alpha : \sigma$

$\Gamma \vdash M: \forall \alpha : \sigma \Rightarrow \Gamma \vdash M \tau : \sigma[\alpha = \tau]$

Пример черчевский нумерал

$\wedge \alpha. \lambda f^{\alpha \rightarrow \alpha}. \lambda x^\lambda.f(fx) : \forall \alpha . (\alpha \rightarrow \alpha) \rightarrow (\alpha \rightarrow \alpha)$

\end{para}

\end{document}