\begin{para}{Преобразование атрибутов при устранении левой рекурсии}

\textbf{Синтезируемые атрибуты}

$A \rightarrow A\alpha, A_0.x = f(A_1.x, \alpha)$

$A \rightarrow \beta, A_0.x = g(\beta)$

~\

Устраним левую рекурсию

$A \rightarrow \beta A', A'.y = g(\beta), A.x = A'.x$

$A' \rightarrow \alpha A', A_2'.y = f(A'_0.y, \alpha), A'_0.x = A'_2.x$

$A' \rightarrow \varepsilon A'.x = A'.y$

~\

\textbf{Наследуемые атрибуты}

$A \rightarrow A\alpha, A_1.x = f(A_0.x, \alpha), \alpha.x = g(A_0.x, A_1.x)$

$A \rightarrow \beta, \beta.x = h(A.x)$

Мы не можем посчитать атрибут беты, потому что во время вызова мы не будем знать глубину предстоящей рекурсии

\end{para}


\begin{para}{Атрибуты при восходящем разборе}

$A \rightarrow \alpha, A.x = f(\alpha)$

Атрибуты терминалов заполняются при считывании.

Синтезируемые атрибуты нетерминалов заполняются очевидно во время свертки



\end{para}